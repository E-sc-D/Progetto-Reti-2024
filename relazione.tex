\documentclass[a4paper,12pt]{report}
\usepackage{alltt, fancyvrb, url}
\usepackage{graphicx}
\usepackage[utf8]{inputenc}
\usepackage{float}
\usepackage{hyperref}
\usepackage[italian]{babel}
\usepackage[italian]{cleveref}
\title{Progetto Programmazione di Reti 
    \\ Chat Client-Server}

\author{Edoardo Scorza \\ Matricola: 010202434}
\date{24 giugno 2024}   
\begin{document}
\maketitle
\tableofcontents
\chapter*{Istruzioni d'uso}
\addcontentsline{toc}{chapter}{Istruzioni d'uso}
La Chat Client-Server è strutturata in 2 file il Client e il Server.
é possibile eseguirli attraverso il file.bat passandogli come 
argomento il numero di client che si vogliono eseguire\\\\
.\textbackslash run 2\\\\
Una volta eseguito il comando verranno generati n + 1 terminali,
con n client e 1 server, l'indirizzo ip e porta del server sono fissati
e verranno mostrati a seguito dell'avvio.
\\\\
Una volta impostato l'ip e la porta del server nel client
verrà richiesto il nome utente, a quel punto è possibile iniziare la conversazione
con i client gia connessi.

\chapter*{Server.py}
\addcontentsline{toc}{chapter}{Server.py}
Il server per il suo funzionamento fa leva\\
su due principali componenti:\\\\
I socket, in questo caso lo stream socket\\
usato per la comunicazione di tipo TCP, ovvero connection oriented\\
affidabile e sicuro per la trasmissione di dati.\\\\
I thread invece permettono al server di gestire piu client\\ in contemporanea
e di separare le sue funzioni, tra cui troviamo:
\begin{itemize}
    \item \texttt{listen\_to\_connections()}: la quale si occupa di ricevere le richieste sulla porta comune e stabilire una nuova porta per la comunicazione fra quel client e il server,
    \item \texttt{Handle\_client(socket)}:listen\_to\_connection creerà un nuovo tread\\ con la funzione Handle\_client che si occuperà di gestire la comunicazione fra il server e il client specifico
\end{itemize}
\chapter*{Client}
\addcontentsline{toc}{chapter}{Client.py}
Il client come il server usa i socket per 
stabilire la comunicazione con il server,\\ a differenza di esso però, non ha bisogno di molti thread,\\
in quanto la sua comunicazione è di tipo 1-1, solo con il server insomma,\\
per questo è anche piu semplice:
\begin{itemize}
    \item \texttt{ Script principale}:\\
Il condice fondamentale che inizializza la GUI e la connessione\\
viene eseguito fuori da una funzione, Non deve essere chiamato ma eseguito come prima cosa,\\in quanto fondamentale per il client
    \item \texttt{send\_messages()}:\\
Questa funzione potremmo dire che è la parte operativa\\ del programma,
si occupa di gestire l'input dell'utente,\\che siano comandi oppure messaggi,
è situato in un altro thread per non causare rallentamenti alla Gui
\end{itemize}
\end{document}